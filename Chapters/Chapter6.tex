\chapter*{Podsumowanie, wnioski i dalsze prace}
\label{chapter:podsumowanie}
\thispagestyle{empty}


W podsumowaniu powinno się znaleźć:
\begin{itemize}
\item{streszczenie metody, która została opisana w pracy (3-4 zdania)},
\item{opis najważniejszych rezultatów wykazujących zrealizowanie celu pracy, proszę podać najlepsze rezultaty,}
\item{wypunktowanie osiągnięć uzyskanych w pracy w kolejności od najważniejszych do mniej ważnych,}
\item{stwierdzenie, ze teza pracy została udowodniona ("Uzyskane rezultaty udowadniają teze pracy ..."),}
\item{krytyczna ocena uzyskanych rezultatów, jeżeli taka sutuacja zachodzi należy przyznać się do elementów, które nie zostały skończone lub wymagają poprawienia, koniecznie trzeba uzasadnić przyczynę tych niedociągnięć.}
\end{itemize}


\subsection*{Dalsze prace związane z tematyką pracy}

\rdm{Dyskusja możliwości kontynuacji badań opisywanych w pracy (około jednej strony).}
Osi�gi aktualnych uk�ad�w SoC pozwalaj� przypuszcza�, �e w nied�ugim czasie ca�o�� prac b�dzie mo�liwa do zaimplementowania w takim uk�adzie. Pozwoli to na zminiaturyzowanie ca�ego systemu praktycznie do wielko�ci stojaka z mikrofonami. 

Mikroprocesor wykorzystywany do tej pory np w multimedialnych telefonach kom�rkowych b�dzie na tyle wydajny, �e pozwoli na umieszczenie wszystkich modu��w potrzebnych do lokalizacji �r�d�a d�wi�ku w jednym uk�adzie. 




