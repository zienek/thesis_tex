\chapter{Matematyczne podstawy}
\label{Rozdzial3}
\lhead{Rozdzia� 3. \emph{Teoretyczny opis rozwi�zania problemu}}

Poni�ej zostanie przedstawiona metodologia bada� wykorzystana podczas projektowania systemu lokalizacji d�wi�ku. 


\rdm{Opis metody, za pomoc� kt�rej rozwi�zano problem w pracy. Powinna to by� prezentacja metody nie nawi�zuj�ca do jej implementacji. Istotne jest wyja�nienie podstaw matematycznych i/lub fizycznych dzia�ania metody.
Przedstawienie algorytmu jako ca�o�ci, zdefiniowanie modu��w i powiązań między modułami. Do modelowania struktury systemu (realizuj�cego metod�) prosz� stosowa� schematy (np. diagramy  UML).
prosz� wyjaśnić nowe poj�cia wykorzystywane w opisie metody.}

\section{Og�lny opis metody}


Najzwyklejsz� metod� obliczenia TDoA jest ta znana pod skr�tem GCC. 


