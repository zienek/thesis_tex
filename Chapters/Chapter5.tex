\chapter{Badania i testy}
\label{chapter:rezultaty}
\thispagestyle{empty}

\section{Opis procedury testowej}

\rdm{W podrozdziale tym należy opisać:}
\begin{itemize}
\item{\rdm{jaki jest cel przeprowadzenia testów (co testy mają wykazać),}}
\item{\rdm{co będzie metryką pozwalającą na ocenę jakości uzyskanych rezultatów (polecane jest stosowanie metryk porównawczych, tzn. np. porównanie jakości obrazu do obrazu wzorcowego),}}
\item{\rdm{jakie testy zostaną przeprowadzone,} }
\item{\rdm{należy wyspecyfikować procedurę testową w stopniu umożliwiającym czytelnikowi jej powtórzenie,}}
\item{\rdm{jakie dane wejściowe zostaną użyte w testach (np. jakie obrazy testowe).}}
\end{itemize}

\section{Prezentacja rezultatów}

\rdm{
\begin{itemize}
\item{Prezentacja rezultatów - tabele z wynikami, wykresy.}
\item{Komentarz do uzyskanych wyników (w szczególności trzeba opisać wyniki najlepsze i najgorsze).}
\item{Wyjaśnienie przyczyny pojawienia się błędnych rezultatów.}
\item{Porównanie rezultatów z danymi wzorcowymi.}
\end{itemize}}

\section{Wnioski}

\rdm{Ogólne wnioski dotyczące uzyskanych rezultatów. Dyskusja jak wypadają uzyskane wyniki na tle danych wzorcowych.}


